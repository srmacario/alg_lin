\documentclass{article}

\usepackage{graphicx}
\usepackage{subcaption}
\usepackage[utf8]{inputenc}
\usepackage{amsmath}
\usepackage{float}

\renewcommand*\contentsname{Sumário}
\begin{document}

\begin{titlepage}
    \begin{center}

        \includegraphics[width=0.4\textwidth]{ime.jpg}

        \vspace{1cm}

        \Huge
        \textbf{Álgebra Linear na Construção da Arquitetura de Controle de Robôs Autônomos com Rodas Omnidirecionais}
 
        \vspace{0.5cm}
        \LARGE
 
        \vspace{1.5cm}
 
        Alexandre Pereira de Freitas\\
        Lucas Rafael de Aguiar Silva\\
        Samuel Morais Barros\\
        Sérgio Reinier Sousa Macário\\
        Vinícius de Freitas Lima Moraes\\
 
        \vfill
 
        \vspace{0.8cm}
 
        \Large
        08/11/2019
 
    \end{center}
\end{titlepage}

\tableofcontents
\newpage

\section{Introdução}
\subsection{Constituição dos robôs com rodas omnidirecionais}

Robôs com movimentação omnidirecional são máquinas capazes de se locomover em todas as direções e ângulos sem a necessidade de rotacionar o seu corpo antes de executar a translação. As peças especiais que permitem isso são as omnirodas, rodas constituídas por várias rodas menores perpendiculares em relação ao plano da roda maior encaixadas ao longo de sua periféria, conforme a figura 1 a seguir:

\begin{figure}[H]
\centering
\includegraphics[width=0.4\textwidth]{ime.jpg}
\caption{Legenda}
\label{Rotulo}
\end{figure}

O princípio de funcionamento dos robôs com omnirodas constitui-se do fato de que, enquanto a roda principal provê tração na direção normal ao eixo do motor, as rodas menores que constituem a roda principal permitem o deslizamento sem atrito (para um estudo ideal) na direção do eixo do motor.

\begin{figure}[H]
\centering
\includegraphics[width=0.4\textwidth]{ime.jpg}
\caption{Legenda}
\label{Rotulo}
\end{figure}


Em geral, utiliza-se três ou quatro rodas omnidirecionais na fabricação de um robô, conforme a figura 2. A composição da tração de cada uma das rodas possibilita a translação e a rotação do robô. O ideal seria utilizar duas rodas omnidirecionais ortogonais, de tal forma que somente a translação fosse permitida, minimizando a complexidade do movimento. Porém, por questões espaciais e de estabilidade, esse arranjo é impossível. 
Nesse ínterim, o ideal das rodas omnidirecionais constitui da busca pela situação na qual as n rodas omnidirecionais empregadas em um robô possibilitam a obtenção da maior velocidade pelo robô no sentido desejado.


\subsection{Fundamentos Físicos}

Para a compreensão do movimento, é necessário analisar tanto a translação quanto a rotação do sistema de acordo com a geometria do robô. Para isso, pode-se modelar o robô dotado de n rodas, tal que:

\begin{figure}[H]
\centering
\includegraphics[width=0.4\textwidth]{ime.jpg}
\caption{Legenda}
\label{Rotulo}
\end{figure}

Para a translação, pode-se aplicar a 2º lei de Newton com relação ao centro do robô:
\[\vec{F}_{resultante} = M\vec{a}_{resultante}\]
\[\vec{a}=\frac{1}{M}\sum_{i=1}^{n}\vec{F}_i\]
Observa-se, por decomposição vetorial, que:
\[a_x=-\frac{1}{M}\sum_{i=1}^{n}F_i\sin\theta_i \text{ (I)}  \]     
\[a_y=\frac{1}{M}\sum_{i=1}^{n}F_i\cos\theta_i \text{ (II)}\]	
Para a rotação, deve-se analisar o torque resultante:
\[\tau_{resultante}=I\dot{\omega}\]
\[\dot{\omega}=\frac{R}{I}\sum_{i=1}^{n}F_i \text{ (III)}\]
O momento de inércia de uma distribuição de massa qualquer pode ser representado por:
\[I=\alpha MR^2\text{ , com } 0\leq\alpha\leq1\]
Desse modo, pode-se simplificar as equações (I), (II) e (III) pela seguinte identidade matricial:
\[
\begin{pmatrix}a_{x}\\a_{y}\\R\dot{\omega} \end{pmatrix}
=\frac{1}{M}
\begin{pmatrix}
  -\sin\theta_1 & -\sin\theta_2 & \cdots & -\sin\theta_n \\
  \cos\theta_1 & \cos\theta_2 & \cdots & \cos\theta_n \\
  \frac{1}{\alpha} & \frac{1}{\alpha} &\cdots & \frac{1}{\alpha}
 \end{pmatrix}
\]
Denomina-se a matrix $3 \times n$ da equação acima de matriz de acoplamento de forças, e representa-se por \( C_\alpha \). \\
Pode-se obter a velocidade final de cada roda bem como a velocidade linear e angular do robô integrando a equação matricial (IV). Porém, é mais interessante analisar a situação em que o robô se apresenta em um espaço Euclidiano, obter a trajetória e, a partir daí, derivar a velocidade de cada roda individualmente.\\
Agrupando as velocidades individuais proporcionadas por cada motor na matriz $m=\begin{pmatrix}v_{1},v_{2},v_{3},v_{4} \end{pmatrix}^T$ e as velocidades euclidianas e a velocidade de rotação do robô na matriz  $v=\begin{pmatrix}v_{1},v_{2},v_{3},v_{4} \end{pmatrix}^T$, nota-se que o movimento das rodas se decompõem em componentes translacionais e rotacionais, segundo a geometria do robô. Nesse panorama, a seguinte equação representa a relação entre a velocidade dos motores individualmente e a velocidades resultantes no robô.
\[
\begin{pmatrix}v_{1}\\ \vdots \\ v_{n} \end{pmatrix}
=
\begin{pmatrix}
  -\sin\theta_1 & \cos\theta_1  & \cdots & 1 \\
  \vdots & \vdots & \cdots  \\
  -\sin\theta_n & \cos\theta_n  &\cdots & 1
 \end{pmatrix}
\]
\[
\begin{pmatrix}v_{x}\\ v_y \\ R\omega \end{pmatrix}
\]
A matriz envolvida no produto matricial acima, bastante semelhante à matriz $C_\alpha$, é chamada matriz de acoplamento de velocidade, e será denotada por D.
Assim, sendo a aceleração do robô representada pelo vetor $a=\begin{pmatrix}v_{1},v_{2},v_{3},v_{4} \end{pmatrix}^T$ e a magnitude (módulo) das forças representada por $f=\begin{pmatrix}v_{1},v_{2},v_{3},v_{4} \end{pmatrix}^T$, podem ser constituídas as seguintes igualdades:
\[a=C_{\alpha}f\]
\[m=Dv\]
Integrando em função de um intervalo de tempo $\Delta t$, obtém-se que $\Delta v=\Delta t \times a$. Assim,
\[\Delta m=\Delta t\times DC_{\alpha}f\]
Os motores possuem um mecanismo que permite a medição da velocidade das rodas em tempo real. Com o fim de controlar o robô, é necessário conhecer o comportamento da matriz v, isto é, é necessário construir uma transformação linear que mapeie os elementos do subespaço gerado pela matriz m nos elementos do subespaço gerado pela matriz v.
Para tal, é necessário inverter a matriz m=Dv. Em geral, isso não é possível por a matriz D não ser quadrada e, portanto, não ser invertível. Porém, pode-se utilizar uma matriz pseudo-inversa pela esquerda, denominada $D^+$, tal que $D^+ D=I_n$. Nota-se, ainda, que $DD^+$ não representa um produto matricial. Nesse ínterim, 
\[D^+m=(D^+D)v=Iv\]
\[v-D^+m\]



\section{Enunciado do problema}
\subsection{Deslizamento de Rodas}

Considere um robô simétrico, com m = $\begin{pmatrix}v_{1},v_{2},v_{3},v_{4} \end{pmatrix}^T$ indicando a velocidade tangencial dos motores,
D a matriz de velocidades, e v um vetor tridimensional $\begin{pmatrix} v_{x},v_{y},Rw \end{pmatrix}^T$.
Dado isso, teste a inconsistência da velocidade dos motores e, consequentemente, se há rodas deslizando.

\subsection{Economia de Energia}

Considerando o resultado do problema anterior, ou seja, caso seja detectado inconsistência com os motores e alguma das rodas esteja deslizando.
É possível corrigir rapidamente esse problema sem alterar a configuração física do robô (número de motores, número de rodas etc.)? Se sim proponha uma solução adequada.

\section{Resolução do problema}
\subsection{Item 1}

Consideremos o robô da figura, no qual os ângulos das rodas superiores são $\alpha$ e os da inferiores são $\beta$, o vetor $m = \begin{pmatrix} v1,v2,v3,v4 \end{pmatrix}^T$ representa
as velocidades tangenciais dos motores e o vetor $v = \begin{pmatrix} v_{x},v_{y},Rw \end{pmatrix}^T$ a velocidade total do robô (translacional e rotacional).
\\ Para procurarmos inconsistências nas velocidades dos motores (que indicam deslizamento nas rodas), partiremos da relação inicial dada por:
\\ \\ $m = Dv$ \\ \\
Como é possível notar, a matriz D não é quadrada, ela não possuui inversa, então não podemos, inicialmente, isolar o vetor "v", no entanto, podemos achar uma matriz equivalente, chamada de Matriz Pseudoinversa ($D^+$), que usaremos com intuito de isolar o vetor das velocidades Euclidianas.
\\ \\ Definição Pseudo-inversa (Moore-Penrose)

De acordo com a definição pseudo-inversa de Moore-Penrose, vemos que se D tem seu espaço coluna LI, podemos calcular a sua inversa da seguinte maneira:
\\ \\ $D^+ = (D^*D)^{-1} D^*$

Demonstração da propriedade:

Daí usando a expressão anterior e multiplicando por D pela direita, temos:
\\ $D^+D = (D^*D)^{-1}(D^*D) = I => (D^+$ é a inversa a esquerda de D)  
\\ \\ Aplicando a pseudo-inversa na equação inicial, obtemos:
$ m = Dv => D^+m = D^+Dv => v = D^+m$
\\ \\ Com a constante comunicação do robô e do controlador acerca da situação das velocidades dos motores, podemos sempre testar a inconsistência da matriz m, de modo que
se as relações $m = Dv$ e $v = D^+m$ são válidas, então $m = DD^+m$ também o é. Para testar a consistência das velocidades, basta testar a validade da última expressão.
\\ Caso essa igualdade não ocorra, podemos afirmar com certeza que alguma roda está com velocidade inconsistente e portanto, deslizando. Contudo, existe a possibilidade de que múltiplas rodas também estejam deslizando em uma taxa que faça com que a relação permaneça válida,
no entando, isto é extremamente improvável.

Assim, vamos trabalhar testando a validade da expressão:
\\ $ m = DD^+m => (I - DD^+m) = 0$
\\ Para $\alpha = \pi/6$ e $\beta = \pi/4$, teremos a seguinte relação:
\\ $(v1 - v2) = (v3 - v4)(sqrt{2/3})$
\\ \\ Logo, se essa relação \textbf{não} for válida, alguma das rodas não está rodando da maneira correta.
\subsection{Item 2}
Para identificarmos uma possível perda ou desperdício de energia precisamos identificar como funciona o vetor aceleração do robô, dado por: 
$ a = C_{\alpha}f_{k}$ Partindo desse ponto, é fácil notar que existem combinações de forças dos motores que geram uma aceleração nula:
$ a = 0 $, como o vetor: $f_{o} = \begin{pmatrix} 1,-1,1,-1\end{pmatrix}$. Em outras palavras, o vetor $f_{o}$ pertence ao núcleo de $C_{\alpha}$.
\\ \\
Essa interessante observação nos permite expandir o raciocínio para resolver o problema de desperdício de energia. Note que se um vetor $g$ pertence ao núcleo de $C_{\alpha}$, então qualquer combinação de vetores $f$ que inclua $g$
produz a mesma aceleração que $f - g$, pois:\\
\\  $ a = C_{\alpha}(f) = C_{\alpha}(f-g) + C_{\alpha}(g) = C_{\alpha}(f-g)$
\\ \\
Como calculado anteriormente na seção 3.1, $dim(C_{alpha}) = 3$, daí, aplicando o teorema do posto-nulidade para $C_{\alpha}$, temos:
\\  \\ $ Rank(C_{\alpha}) + Ker(C_{\alpha}) = Dim(coluna)$ \\ \\
Daí tiramos que $Ker(C_{\alpha}) = 1 $, e portanto qualquer vetor no núcleo de $C_{\alpha}$ é da forma $\lambda f_{k}$.
\\ O fato interessante de se notar é que primeiro, testamos as validades das velocidades através da equação: $(I-DD^+)m = 0$
e agora, concluímos que qualquer vetor no núcleo de $C_{\alpha}$ gera uma aceleração nula no robô. Podemos estreitar ainda mais essa relação,
se usarmos o conceito de operador de projeção ortogonal:
\\ \\ As matrizes $DD^+ e D^+D$ são operadores de projeção ortogonal, $P$ , ou seja, são hermitianos ($P = P^*$), por definição, e são idempotentes
($P^2 = P$). Daí, as seguintes propriedades se seguem:
\\ $DD^+$ é o operador de projeção ortogonal na $Im(D)$, e por consequência, $I - DD^+$ é o operador de projeção ortogonal em $ker(A)$.
\\ \\ Se analisarmos as matrizes $C_{\alpha}$ e $D$, podemos notar que $C_{\alpha} = (1/{\alpha}M)(D^T)$, como nossas matrizes estão no corpo dos $R$, sabemos que $D^T = D^*$,
então podemos ver que $C_{\alpha}$ é a matriz adjunta de $D$ multiplicada por uma constante!
\\ Então, se quisermos corrigir o deslizamento de velocidades, verificando a consistência da equação $(I-DD^+)m = 0$, basta vermos que o operador $(I - DD^+)$ projeta $m$ no $ker(D^*)$,
logo a relação que queremos manter válida é que $m$ seja \textbf{ortogonal} ao vetor que compõem a base do $ker(C_{\alpha})$.
\\ \\ Correção:

Podemos visualizar a correção de deslizamento, identificando o espaço vetorial da transformação $(I - DD^+)$.
Nesse espaço, de dimensão = 4, existe um subespaço tridimensional de valores consistentes, ou seja, que não deslizam. Ortogonalmente à esse subespaço,
possuímos o vetor $f_{k}$, sempre que as rodas estão deslizando, estamos gastando energia, pois o vetor das velocidades contém uma componente na direção $f_{k}$.
Daí, a nossa correção para velocidades inconsistentes gera, consequentemente, uma economia de energia para o sistema do robô.
\\ \\ Além disso, quando mapeamos as acelerações Euclidianas $a$ com as forças dos motores $f$, sempre obtemos resultados consistentes, através da expressaõ:
\\ $f = C_{\alpha}^+a$
\\ Veja que $rank(C_{\alpha}) = 3$, então pelo teorema do posto-nulidade temos que: $ker(C_{\alpha}) = 0$. $C_{\alpha}$, portanto, mapeia as acelerações $a$  em $f$ (subespaço de 3 dimensões).
Se esse espaço possuísse algum elemento $u$ do núcleo de $C_{\alpha}$, então deveria existir uma aceleração $a$, diferente de 0, tal que $u = C_{\alpha}^+a$. Como vimos anteriormente que $c_ {\alpha}u = a = 0$, chegaríamos a um absurdo.

\section{Conclusão}

\end{document}documentclass{article}

\usepackage{grapicx}
\usepackage[utf8]{inputenc}

\renewcommand*\contentsname{Sumário}
\begin{document}

\begin{figure}
    \includegraphics[width=\linewidth]{ime.jpg}
\end{figure}

\tableofcontents
\newpage

\section{Introdução}

\section{Discussão do Problema}

\section{Resolução do Problema}

\section{Conclusão}

\end{document}