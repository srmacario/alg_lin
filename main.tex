\documentclass{article}

\usepackage{graphicx}
\usepackage{subcaption}
\usepackage[utf8]{inputenc}
\usepackage{amsmath}

\renewcommand*\contentsname{Sumário}
\begin{document}

\begin{titlepage}
    \begin{center}

        \includegraphics[width=0.4\textwidth]{ime.jpg}

        \vspace{1cm}

        \Huge
        \textbf{Álgebra Linear na Construção da Arquitetura de Controle de um Robô Autônomo com Rodas Ominidirecionais}
 
        \vspace{0.5cm}
        \LARGE
 
        \vspace{1.5cm}
 
        Alexandre Pereira de Freitas\\
        Lucas Rafael de Aguiar Silva\\
        Samuel Morais Barros\\
        Sérgio Reinier Sousa Macário\\
        Vinícius de Freitas Lima Moraes\\
 
        \vfill
 
        \vspace{0.8cm}
 
        \Large
        08/11/2019
 
    \end{center}
\end{titlepage}

\tableofcontents
\newpage

\section{Introdução}

\section{Enunciado do problema}
\subsection{Deslizamento de Rodas}

Considere um robô simétrico, com m = $\begin{pmatrix}v_{1},&v_{2},&v_{3},&v_{4} \end{pmatrix}^T$ indicando a velocidade tangencial dos motores,
D a matriz de velocidades, e v um vetor tridimensional $\begin{pmatrix} v_{x},&v_{y},&Rw \end{pmatrix}^T$.
Dado isso, teste a inconsistência da velocidade dos motores e, consequentemente, se há rodas deslizando.

\subsection{Economia de Energia}

Considerando o resultado do problema anterior, ou seja, caso seja detectado inconsistência com os motores e alguma das rodas esteja deslizando.
É possível corrigir rapidamente esse problema sem alterar a configuração física do robô (número de motores, número de rodas etc.)? Se sim proponha uma solução adequada.

\section{Resolução do problema}
\subsection{Item 1}
\subsection{Item 2}
Para identificarmos uma possível perda ou desperdício de energia precisamos identificar como funciona o vetor aceleração do robô, dado por: 
$ a = C_{\alpha}f_{k}$ Partindo desse ponto, é fácil notar que existem combinações de forças dos motores que geram uma aceleração nula:
$ a = 0 $, como o vetor: $f_{o} = \begin{pmatrix} 1,&-1,&1,&-1\end{pmatrix}$. Em outras palavras, o vetor $f_{o}$ pertence ao núcleo de $C_{\alpha}$.
\\ \\
Essa interessante observação nos permite expandir o raciocínio para resolver o problema de desperdício de energia. Note que se um vetor $g$ pertence ao núcleo de $C_{\alpha}$, então qualquer combinação de vetores $f$ que inclua $g$
produz a mesma aceleração que $f - g$, pois:\\
\\  $ a = C_{\alpha}(f) = C_{\alpha}(f-g) + C_{\alpha}(g) = C_{\alpha}(f-g)$
\\ \\
Como calculado anteriormente na seção 3.1, a dimensão
Aplicando o teorema do posto-nulidade para $C_{\alpha}$, temos:
\\  \\ $ Rank(C_{\alpha}) + Ker(C_{\alpha}) = Dim(coluna)$ \\ \\
Daí tiramos que $Ker(C_{\alpha}) = 1 $, e portanto qualquer vetor no núcleo de $C_{\alpha}$ é da forma $\lambda f_{k}$

\section{Conclusão}

\end{document}documentclass{article}

\usepackage{grapicx}
\usepackage[utf8]{inputenc}

\renewcommand*\contentsname{Sumário}
\begin{document}

\begin{figure}
    \includegraphics[width=\linewidth]{ime.jpg}
\end{figure}

\tableofcontents
\newpage

\section{Introdução}

\section{Discussão do Problema}

\section{Resolução do Problema}

\section{Conclusão}

\end{document}